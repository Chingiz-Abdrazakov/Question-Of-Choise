\documentclass[14 pt]{extreport}

\usepackage[T2A]{fontenc}
\usepackage[utf8]{inputenc}
\usepackage[english, russian]{babel}
\usepackage{graphicx}
\usepackage{amsmath, amsfonts, amssymb, mathtools}

\renewcommand{\rmdefault}{cmss}
\renewcommand{\ttdefault}{cmss}
\usepackage{euler}

\def\Z{\mathbb{Z}}
\def\Q{\mathbb{Q}}
\def\C{\mathbb{C}}
\def\R{\mathbb{R}}
\def\T{\mathbb{T}}
\def\F{\mathbb{F}}
\def\cd{\cdot}
\def\Fi{\varphi}
\def\Lra{\Leftrightarrow}
\def\om{\omega}

\author{Чингиз Абдразаков}
\title{Вопрос по выбору по Термодинамике\\На тему "Явление Лейденфроста"}
\date{}

\begin{document}
	\maketitle
	
	\newpage
	\section*{Теоретическая часть}
	
	\textbf{Эффект Лейденфроста} -- явление, при котором жидкость в контакте с твердой поверхностью, температура которой значительно превышает температуру кипения этой жидкости, образует теплоизолирующую прослойку пара между поверхностью и жидкостью, замедляющую быстрое выкипание, например, капли жидкости на этой поверхности. \\\\
	
	Это явление можно довольно наглядно наблюдать и в домашних условиях. В случае с водой этот эффект можно наблюдать, если капать ее на горячую сковороду по мере нагревания последней. Что мы наблюдаем? 
	
	При температурах ниже 100 \textdegree C капельки воды просто растекаются по поверхности и постепенно испаряются. После же достижения температуры кипения капли воды будут испаряться с характерным шипением и довольно быстро. 
	
	Далее, после того как температура превысила точку Лейденфроста, начинает проявляться эффект: при попадании на сковороду капли собираются в небольшие шарики и "бегают" по ней -- вода не выкипает на сковороде значительно дольше, чем при более низких температурах. \\\\
	
	Основная причина тому -- при температурах выше точки Лейденфроста нижняя часть капли мгновенно испаряется при контакте с горячей поверхностью. Получившаяся прослойка пара подвешивает оставшуюся часть капли над ней, предотвращая прямое соприкосновение с горячим телом. А в силу того, что теплопроводность пара значительно ниже, если сравнивать с той же жидкостью, то теплообмен между непосредственно капелькой и горячей поверхностью заметно замедляется, что позволяет капле "левитировать" на прослойке газа.\\\\
	
	Температуру Лейденфроста довольно сложно предсказать заранее, так как она зависит от свойств поверхности и примесей в жидкости. Одна из довольно приблизительных оценок дает значение точки Лейденфроста для воды на сковороде в 193 \textdegree C.\\\\
	
	Теперь же оценим время жизни полушаровой капли жидкости, находящейся над очень тонким слоем пара.
	\begin{center}
		\includegraphics[width=1\linewidth]{Leidenfrost_1}
	\end{center}
	
	Будем считать, что поток пара из-под капли является ламинарным и ведет себя как Ньютоновская жидкость с коэффициентом вязкости $ \eta $ и температурной проводимостью $ \kappa $. Скрытая теплота испарения жидкости равняется $ l $. Для Ньютоновской жидкости напряжение сдвига 
	$\dfrac{F}{A} = \eta \dfrac{dv}{dz}$, где $ \dfrac{dv}{dz} $ -- изменение скорости потока $ v $ вдоль направления $ z $. $ z $ -- это расстояние перпендикулярно к направлению потока, а направление $ F $ касательно к поверхности, имеющей площадь $ A $.
	
	\begin{center}
		\includegraphics[width=1.2\linewidth]{Leidenfrost_2}
	\end{center}
	
	$ v $ -- скорость потока пара в радиальном направлении по высоте $ z $, отсчитываемой от середины слоя пара. Давление пара $ P $ повышается при приближении к центру $ O $. Это приводит к тому, что возникает исходящий поток пара и возникает сила, которая удерживает каплю против силы тяжести. Толщина слоя пара при условии механического и температурного равновесия равняется $ b $. Для Ньютоновского потока пара имеет место следующее равенство:
	\[\dfrac{d}{dz}v = \dfrac{z}{\eta} \dfrac{d}{dr}P\]
	
	Вывод:
	
	Записываем условие механического равновесия для кубика пара на некотором расстоянии $ r $ от центра. Пусть его высота равна $ z $, длина $ dr $, ширина $ dy $. Тогда
	\[P(r)S + F_{v} = P(r + dr)S,~~~S = dy \cd z, ~~~ F_{v} = \eta \dfrac{dv}{dz}A, ~~~ A = dy \cd dr\]
	\[dP \cd dy \cd z = \eta \dfrac{dv}{dz} dy \cd dr\]
	\[\dfrac{dv}{dz} = \dfrac{z}{\eta}\dfrac{dP}{dr}\]\\
	
	Полагая, что при $ z = \pm \dfrac{b}{2} ~~$ $ v(\pm b/2) = 0 $, получаем:
	
	\[\int_{0}^{v}dv = \dfrac{1}{\eta} \dfrac{dP}{dr} \int_{b/2}^{z}zdz\]
	\[v(z) = \dfrac{1}{2\eta} \dfrac{dP}{dr}\left( z^{2} - \dfrac{b^{2}}{4} \right) \]
	
	Теперь положим $ Q $ скорость истечения объема пара из-под капли через цилиндрическую поверхность радиуса $ r $ и высоты $ b $. Тогда для объемной скорости истечения пара через слой высоты $ \delta z $ имеем:
	
	\[\delta Q = v(z) \cd 2\pi r \delta z\]
	
	Подставляя выражение для скорости в формулу, получаем:
	
	\[Q = 2 \int_{z = 0}^{b/2} v(z) \cd 2\pi r dz = \left( \dfrac{2\pi r}{\eta} \dfrac{dP}{dr} \right) \int_{z = 0}^{b/2}\left[ z^{2} - \dfrac{b^{2}}{4} \right]dz  \]
	Откуда
	\[Q = -\dfrac{\pi r b^{3}}{6\eta} \dfrac{dP}{dr}\]
	
	Далее обозначим плотность пара, который образуется при контакте с горячей поверхностью, через $ \rho_{v} $.
	
	По закону Фурье количество теплоты, передаваемое поверхности капли площадью $ \pi r^{2} $ и разностью температур $ \varDelta T $ (относительно нагретой поверхности), через слой пара толщиной $ b $ и теплопроводностью $ \kappa $ в единицу времени равно
	\[\dfrac{dq}{dt} \approx \dfrac{\pi r^{2} \kappa \varDelta T}{b}\]
	
	Полагая, что тепло, передаваемое от нагретой поверхности, расходуется на испарение капли, имеем равенство:
	\[\rho_{v}Ql = \dfrac{\pi r^{2} \kappa \varDelta T}{b}\]
	
	Подставим выражение для теплоты, откуда:
	
	\[\dfrac{dP}{dr} = -\left( \dfrac{6\eta \kappa \varDelta T}{\rho_{v}lb^4} \right) \cd r \]
	
	Считая, что при $ r = R $ давление паров равно атмосферному $ P_{a} $, имеем
	\[\int_{P_{a}}^{P(r)}dP = - \left( \dfrac{6\eta \kappa \varDelta T}{\rho_{v}lb^4} \right) \int_{R}^{r}dr\]
	\[P(r) = P_{a} + \left( \dfrac{3\eta \kappa \varDelta T}{\rho_{v}lb^4} \right)(R^{2} - r^{2})\]
	
	Тогда результирующая сила, удерживающая каплю, равна:
	
	\[f = \int_{r = 0}^{R}[P(r) - P_{a}] 2\pi rdr = \dfrac{3\pi \eta \kappa \varDelta T R^{4}}{2\rho_{v}lb^{4}}\]
	
	В силу формы капли ее масса 
	\[m = \dfrac{2\pi}{3}R^{3} \rho_{0}\]
	
	Получаем
	\[\dfrac{2\pi}{3}R^{3} \rho_{0} g = \dfrac{3\pi \eta \kappa \varDelta T R^{4}}{2\rho_{v}lb^{4}}\]
	
	Получим выражение для $ b $:
	
	\[b = \sqrt[4]{\dfrac{9 \eta \kappa R \varDelta T}{4 \rho_{0} \rho_{v} l g}}\]
	
	Подставим это выражение в выражение для давления:
	\[P(r) = P_{a} + \left( \dfrac{4}{3}\dfrac{\rho_{0}g}{R} \right)(R^{2} - r^{2}) \]
	Откуда
	\[\dfrac{d}{dr}P(r) = -\left( \dfrac{8}{3}\dfrac{\rho_{0}g}{R} \right)r\]
	
	Тогда время жизни может быть найдено из следующего соотношения:
	\[\dfrac{d}{dt}\left( \dfrac{2}{3}\pi R^{3}\rho_{0} \right) = -Q\rho_{v} = -\beta R^{7/4}\]
	
	Где
	\[Q\rho_{v} = \left( \dfrac{2\pi b^{3}R}{12\eta} \right) \left( \dfrac{8}{3}\dfrac{\rho_{0}g}{\rho_{v}} \right) R \rho_{v}
	 = \left( \dfrac{4\pi \rho_{v} \rho_{0} g R}{9\eta} \right)b^{3} =  \]
	 \[= \left(\dfrac{4\pi^{4}\kappa^{3}\rho_{v} \rho_{0} g (\varDelta T)^{3}}{9\eta l^{3}}\right)^{1/4} \cd R^{7/4} 
	 = \beta R^{7/4} \]
	 
	 Итак, получаем
	 \[\int_{R}^{0}R^{1/4}dR = -\int_{0}^{\tau}\dfrac{\beta}{2\pi \rho_{0}}dt\]\\\\
	 
	 Время жизни капли определяется соотношением:
	 \[\tau = \dfrac{8}{5}\left( \dfrac{9\eta \rho_{0}^{3}l^{3}}{4\kappa^{3} \rho_{v} g (\varDelta T)^{3}} \right)^{1/4}\cd R^{5/4} \]\\\\\\
	 
	 
	 \section*{Экспериментальная часть}
	 
	 Сперва проверим, верна ли полученная формула в точке Лейденфроста. В ней формулу для времени жизни капли можно представить в следующем виде:
	 \[\tau = \alpha R^{5/4}\]
	 
	 Точка Лейденфроста была зафиксирована экспериментально, когда время испарения капель было максимальным. Ввиду невысокой мощности используемой нагревательной плитки и тепловых потерь, температура изменялась достаточно медленно, что и позволило провести измерения. Размеры капель фиксировались по снимкам, которые производились в течение измерений, поэтому являются приближенными. 
	 
	 \begin{tabular} {| c | c |}
	 	\hline
	 	$R$, мм & $ \tau $, с\\ \hline 
	 	1 & 40 \\ \hline 
	 	1.5 & 75 \\ \hline 
	 	2 & 118 \\ \hline 
	 	2.5 & 147 \\ \hline 
	 	3 & 190  \\ \hline 
	 	3.5 & 223 \\ \hline 
	 	4 & 265 \\ \hline 
	 \end{tabular}
 
 	
 	\begin{center}
 		\includegraphics[width=1.2 \linewidth]{Leidenfrost_3}
 	\end{center}
 	
 	
 	
 	
 	
 	Откуда по МНК вычисляем угол наклона:
 	\[\alpha = (2.7 \pm 0.4) \cd 10^{-5} \dfrac{\text{с}}{\text{м}^{5/4}}\]
 	
 	Откуда находим $ \varDelta T = \dfrac{8}{5}\left( \dfrac{18\eta \rho_{0}^{3}l^{3}}{5k^{3}\rho_{v} g \alpha^{4}} \right)^{1/3} \approx 172 \text{\textdegree C}  $\\
 	
 	Полагая температуру воды порядка комнатной температуры $ T_0 \approx 20 $\textdegree C, находим температуру Лейденфроста.
 	\[T_{L} \approx (192 \pm 41)\text{\textdegree C} \]
 	
 	При вычислениях были использованы следующие величины: $ \eta = 1.2 \cd 10^{-5} $ Па $ \cd $ с; $ \rho_{0} = 1000 $ кг/м$^{3}$; $ l = 2.26 \cd 10^{6} $ Дж/кг, $ \kappa = 0.024 $ Вт/(м $ \cd $ \textdegree C); $\rho_{v} = 0.598$ кг/м$ ^{3} $. 
 	
 	\begin{center}
 		\includegraphics[width=0.7\linewidth]{Leidenfrost_4}
 	\end{center}
 	
 	\begin{center}
 		\includegraphics[width=0.7\linewidth]{Leidenfrost_5}
 	\end{center}
 	
 	
 	
 	
	 
	
	
	
	
	
	
	
	
	
\end{document}